\documentclass{scrartcl}
\usepackage[utf8]{inputenc}
\usepackage[english, 53.02]{proposal}

\addbibresource{sources.bib}

\newcommand{\spokesperson}{Dr.\ Max Mustermann, Musterstadt}
\newcommand{\project}{Title of the Research Unit}

\begin{document}

{\raggedright{} \normalsize \bfseries
	Overall Description of the Research Unit and the Coordination Proposal \par
    \project{} \par
    \spokesperson{} \par
	\rule{\textwidth}{0.5pt} \par
	Overall Description of the Research Unit and the Coordination Proposal
}

\section{State of the art and preliminary work}

\subsection{For a renewal proposal: Report on the progress to date}

\subsection{Project-related publication}
\printbibliography[heading=none]


\section{Objectives and joint work programme}

\subsection{Objectives of the overall project and expected benefits of collaboration within the unit, incl.\ a description of the group composition and their project-specific qualifications}

\subsection{Joint work programme including proposed research methods}

\subsection{Research data and knowledge management}

\subsection{Potential impact on the research area and local research environment}

\subsection{Measures to advance research careers}

\subsection{National and international cooperation and networking}

\subsection{Collaboration with international cooperation partners}

\subsection{Description of the spokesperson's qualifications }

% For Clinical Research Units. If you are applying for a Clinical Research Unit, please complete items 2.9 through 2.12. If you are applying or a Research Unit, please delete these items.
\subsection{Scientific qualifications of the head of the Research Unit}

\subsection{How does the Clinical Research Unit contribute to the research profile of the university/department of medicine? }

\subsection{Expected benefits of collaboration between clinicians and basic researchers }

\subsection{Modalities of performance-based funding allocations for research and teaching through the department of medicine}


\section{Coordination}

\subsection{Description of how joint objectives and the joint work programme will be implemented in the coordination project}

\subsection{Requested modules}

\subsubsection{Coordination Module}

\subsubsection{Network Funds Module (Funding for Staff, Direct Project Costs and Instrumentation)}

\subsubsection{Start-Up Funding Module}

\subsubsection{Gender Equality Measures in Research Networks Module}

\subsubsection{Professorship Module}

\subsubsection{Temporary Substitute for Clinician Module}

\subsubsection{Project-Specific Workshop Module}

\subsubsection{Mercator Fellow Module}

\subsubsection{Public Relations Module}


\section{Project requirements}

\subsection{Employment status information}

\subsection{Composition of the project group}

\subsection{Researchers with whom you have agreed to cooperate on \underline{this} project}

\subsection{Scientific equipment}

\subsection{Project-relevant cooperation with commercial enterprises}

\subsection{Project-relevant participation in commercial enterprises}


\section{Additional Information}

\end{document}
