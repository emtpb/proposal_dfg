\documentclass{scrartcl}
\usepackage[utf8]{inputenc}
\usepackage[english, 53.02]{proposal}

\addbibresource{sources.bib}

\newcommand{\spokesperson}{Dr.\ Max Mustermann, Musterstadt}
\newcommand{\project}{Title of the Research Unit}

\begin{document}

{\raggedright{} \normalsize \bfseries
	Overall Description of the (Clinical) Research Unit and the Coordination Proposal \par
    \project{} \par
    \spokesperson{} \par
	\rule{\textwidth}{0.5pt} \par
	Overall Description of the (Clinical) Research Unit and the Coordination Proposal
}

\section{State of the art and preliminary work}

\subsection{For a renewal proposal: Report on the progress to date}


\section{Objectives and joint work programme}

\subsection{Objectives of the overall project and expected benefits of collaboration within the unit, incl.\ a description of the group composition and their project-specific qualifications}

\subsection{Joint work programme including proposed research methods}

\subsection{Handling of research data}

\subsection{Potential impact on the research area and local research environment; distinction from other ongoing programmes directly related to the research topic}

\subsection{Description of the spokesperson's qualifications}

\subsection{Promotion of researchers in early career phases}

\subsection{Promotion of equity and diversity}

\subsection{National and international cooperation and networking}

\subsection{Collaboration with international cooperation partners}


% For Clinical Research Units. If you are applying for a Clinical Research Unit, please complete items 2.10 through 2.13. If you are applying or a Research Unit, please delete these items.
\subsection{Scientific qualifications of the head of the Research Unit}

\subsection{How does the Clinical Research Unit contribute to the research profile of the university/department of medicine?}

\subsection{Expected benefits of collaboration between clinicians and basic researchers}

\subsection{Contribution by the department of medicine}


\section{Project- and subject-related list of publications}
\printbibliography[heading=none]

\clearpage
\section{Coordination}

\subsection{Description of how joint objectives and the joint work programme will be implemented in the coordination project}

\subsection{Requested modules}

\subsubsection{Coordination Module}

\subsubsection{Module Network Funds (Funding for Staff, Direct Project Costs and Instrumentation)}

\subsubsection{Module Start-Up Funding}

\subsubsection{Module Standard Allowance for Equity and Diversity}

\subsubsection{Module Professorships}

\subsubsection{Module Temporary Substitute for Clinicians}

\subsubsection{Module Project-Specific Workshops}

\subsubsection{Module Mercator Fellow}

\subsubsection{Module Public Relations}


\section{Project requirements}

\subsection{Employment status information}

\subsection{Composition of the project group}

\subsection{Researchers with whom you have agreed to cooperate on this project}

\subsection{Scientific equipment}

\subsection{Project-relevant cooperation with commercial enterprises}

\subsection{Project-relevant participation in commercial enterprises}

\subsection{Researchers with whom you have collaborated scientifically within the past three years}

\subsection{Considerations on aspects of ecological sustainability in the planning and implementation of the project}


\section{Other Information}

\end{document}
