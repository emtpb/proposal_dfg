\documentclass[german, 53.01]{proposal}

\addbibresource{sources.bib}

\applicant{Dr.\ Max Mustermann, Musterstadt}
\project{Titel des Projekts}

\begin{document}

\section{Ausgangslage}

\subsection*{Stand der Forschung und eigene Vorarbeiten}


\section{Ziele und Arbeitsprogramm}

\subsection{Voraussichtliche Gesamtdauer des Projekts}

\subsection{Ziele}

\subsection{Arbeitsprogramm inkl.\ vorgesehener Untersuchungsmethoden}

\subsection{Umgang mit Forschungsdaten}

\subsection{Relevanz von Geschlecht und/oder Vielfältigkeit}


\section{Projekt- und themenbezogenes Literaturverzeichnis}

\printbibliography[heading=none]

\clearpage
\renewcommand{\maxpage}{8}
\setcounter{page}{1}

\section{Begleitinformationen zum Forschungskontext}

\subsection{Angaben zu ethischen und/oder rechtlichen Aspekten des Vorhabens}

\subsubsection{Allgemeine ethische Aspekte}

\subsubsection{Erläuterungen zu den vorgesehenen Untersuchungen am Menschen, an vom Menschen entnommenem Material oder mit identifizierbaren Daten}

\subsubsection{Erläuterungen zu den vorgesehenen Untersuchungen bei Versuchen an Tieren}

\subsubsection{Erläuterungen zu Forschungsvorhaben an genetischen Ressourcen (oder darauf bezogenem traditionellem Wissen) aus dem Ausland}

\subsubsection{Erläuterungen zu möglichen sicherheitsrelevanten Aspekten}

\subsubsubsection{\glqq{}Dual-Use Research of Concern\grqq{}; Außenwirtschaftsrecht}

\subsubsubsection{Risiken in internationalen Kooperationen}

\subsubsection{Reflexion zu ökologischen Nachhaltigkeitsaspekten in der Planung und Durchführung des Vorhabens}

\subsection{Angaben zur Dienststellung}

\subsection{Angaben zur Erstantragstellung}

\subsection{Zusammensetzung der Projektarbeitsgruppe}

\subsection{Zusammenarbeit mit Wissenschaftler*innen in Deutschland in diesem Projekt}

\subsection{Zusammenarbeit mit Wissenschaftler*innen im Ausland in diesem Projekt}

\subsection{Wissenschaftler*innen, mit denen in den letzten drei Jahren wissenschaftlich zusammengearbeitet wurde}

\subsection{Projektrelevante Zusammenarbeit mit erwerbswirtschaftlichen Unternehmen}

\subsection{Projektrelevante Beteiligungen an erwerbswirtschaftlichen Unternehmen}

\subsection{Apparative Ausstattung}

\subsection{Weitere Antragstellungen}

\subsection{Weitere Angaben}


\section{Beantragte Module/Mittel}

\subsection{Basismodul}

\subsubsection{Personalmittel}
\begin{funds}[Personalmittel]

\positionmul{Wissenschaftliche:r Mitarbeiter:in, TV-L 13, 36 Personalmonate}{5375}{36}

\end{funds}

\subsubsection{Sachmittel}
\begin{funds}[Sachmittel]

\subsubsubsection{Geräte bis 10.000\,\euro, Software und Verbrauchsmaterial}

\position{Platinenanfertigung, elektrische Bauelemente, Leitungen, etc.}{3000}

\subsubsubsection{Reisemittel}

\subsubsubsection{Mittel für wissenschaftliche Gäste \textnormal{(ausgenommen Mercator-Fellow)}}

\subsubsubsection{Mittel für Versuchstiere}

\subsubsubsection{Sonstige Mittel}

\subsubsubsection{Publikationsmittel}

\end{funds}

\subsubsection{Investitionsmittel}

\subsubsubsection{Geräte über 10.000\,\euro}

\subsubsubsection{Geräte über 50.000\,\euro}

\subsection{Modul Eigene Stelle}

\subsection{Modul Vertretung}

\subsection{Modul Rotationsstellen}

\subsection{Modul Mercator Fellow}

\subsection{Modul Projektspezifische Workshops}

\subsection{Modul Öffentlichkeitsarbeit}

\subsection{Modul Pauschale für Chancengleichheitsmaßnahmen}

\end{document}
