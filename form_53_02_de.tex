\documentclass[german, 53.02]{proposal}

\addbibresource{sources.bib}

\applicant{Dr.\ Max Mustermann, Musterstadt}
\project{Titel der Forschungsgruppe}

\begin{document}

\section{Stand der Forschung und eigene Vorarbeiten}

\subsection{Bei einem Fortsetzungsantrag: Bericht über die bisherigen Arbeiten}

\section{Ziele der (Klinischen) Forschungsgruppe und gemeinsames Arbeitsprogramm}

\subsection{Ziele des Gesamtprojektes und erwarteter Gewinn durch die Zusammenarbeit im Verbund, u.\ a.\ Erläuterung der Zusammensetzung der Gruppe und deren spezifischer Qualifikation}

\subsection{Gemeinsames wissenschaftliches Arbeitsprogramm inkl.\ vorgesehener Untersuchungsmethoden}

\subsection{Umgang mit Forschungsdaten}

\subsection{Erläuterung der Schwerpunktsetzung und potenziellen Wirkung im Fachgebiet und ggf.\ vor Ort; Abgrenzung zu anderen laufenden Programmen mit direktem thematischem Bezug}

\subsection{Erläuterung zur Qualifikation des*der Sprecher*in}

\subsection{Förderung von Wissenschaftler*innen in frühen Karrierephasen}

\subsection{Förderung von Chancengleichheit}

\subsection{Nationale und ggf.\ internationale Kooperation und Vernetzung}

\subsection{Projektbeteiligung von Kooperationspartner*innen im Ausland}

% Für Klinische Forschungsgruppen. Bitte löschen Sie die Punkte 2.10 bis 2.13, falls Sie einen Antrag im Rahmen einer Forschungsgruppe stellen.
\subsection{Erläuterung zur Qualifikation des*der Leiter*in}

\subsection{Wie trägt die Klinische Forschungsgruppe zur wissenschaftlichen Profilbildung der Hochschule bzw. Medizinischen Fakultät bei?}

\subsection{Erwarteter Gewinn durch die Zusammenarbeit zwischen Klinikerinnen bzw. Klinikern und Grundlagenwissenschaftlerinnen bzw. Grundlagenwissenschaftlern}

\subsection{Erläuterung des Beitrags der Medizinischen Fakultät}


\section{Projekt- und themenbezogenes Literaturverzeichnis}
\printbibliography[heading=none]


\section{Koordination}

\subsection{Erläuterung der Umsetzung der gemeinsamen Ziele und des gemeinsamen Arbeitsprogramms im Koordinationsprojekt}

\subsection{Beantragte Module}

\subsubsection{Modul Koordinierung}

\subsubsection{Modul Verbundmittel (Personal-, Sach- und Investitionsmittel)}

\subsubsection{Modul Anschubförderung}

\subsubsection{Modul Pauschale für Chancengleichheitsmaßnahmen}

\subsubsection{Modul Professur}

\subsubsection{Modul Rotationsstellen}

\subsubsection{Modul Projektspezifische Workshops}

\subsubsection{Modul Mercator-Fellow}

\subsubsection{Modul Öffentlichkeitsarbeit}


\section{Voraussetzungen für die Durchführung des Vorhabens}

\subsection{Angaben zur Dienststellung}

\subsection{Zusammensetzung der Projektarbeitsgruppe}

\subsection{Wissenschaftler*innen, mit denen für \underline{dieses} Vorhaben eine konkrete Vereinbarung zur Zusammenarbeit besteht}

\subsection{Apparative Ausstattung}

\subsection{Projektrelevante Zusammenarbeit mit erwerbswirtschaftlichen Unternehmen}

\subsection{Projektrelevante Beteiligungen an erwerbswirtschaftlichen Unternehmen}

\subsection{Wissenschaftler*innen, mit denen in den letzten drei Jahren wissenschaftlich zusammengearbeitet wurde}

\subsection{Reflexion zu ökologischen Nachhaltigkeitsaspekten in der Planung und Durchführung des Vorhabens}


\section{Weitere Angaben}

\end{document}
