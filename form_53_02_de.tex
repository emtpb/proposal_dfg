\documentclass{scrartcl}
\usepackage[utf8]{inputenc}
\usepackage[german]{proposal}

\addbibresource{sources.bib}

\newcommand{\spokesperson}{Dr.\ Max Mustermann, Musterstadt}
\newcommand{\project}{Titel der Forschungsgruppe}

\begin{document}

{\raggedright{} \normalsize \bfseries
	Gesamtbeschreibung der Forschungsgruppe und des Koordinationsantrags \par
    \project{} \par
    \spokesperson{} \par
	\rule{\textwidth}{0.5pt} \par
	Gesamtbeschreibung der Forschungsgruppe und des Koordinationsantrags
}

\section{Stand der Forschung und eigene Vorarbeiten}

\subsection{Bei einem Fortsetzungsantrag: Bericht über die bisherigen Arbeiten}

\subsection{Projektbezogenes Publikationsverzeichnis}
\printbibliography[heading=none]


\section{Ziele der Forschungsgruppe und gemeinsames Arbeitsprogramm}

\subsection{Ziele des Gesamtprojektes und erwarteter Gewinn durch die Zusammenarbeit im Verbund, u.\ a.\ Erläuterung der Zusammensetzung der Gruppe und deren spezifischer Qualifikation}

\subsection{Gemeinsames wissenschaftliches Arbeitsprogramm inkl.\ vorgesehener Untersuchungsmethoden}

\subsection{Forschungsdaten- und Wissensmanagement im Verbund}

\subsection{Erläuterung der Schwerpunktsetzung und potenziellen Wirkung im Fachgebiet und ggf.\ vor Ort}

\subsection{Maßnahmen zur Förderung wissenschaftlicher Karrieren}

\subsection{Nationale und ggf.\ internationale Kooperation und Vernetzung}

\subsection{Projektbeteiligung von Kooperationspartnerinnen und -partnern im Ausland}

\subsection{Erläuterung zur Qualifikation des Sprechers bzw.\ der Sprecherin}

% Für Klinische Forschungsgruppen. Bitte löschen Sie die Punkte 2.9 bis 2.12, falls Sie einen Antrag im Rahmen einer Forschungsgruppe stellen.
\subsection{Erläuterung zur Qualifikation des Leiters bzw.\ der Leiterin}

\subsection{Wie trägt die Klinische Forschungsgruppe zur wissenschaftlichen Profilbildung der Hochschule bzw. Medizinischen Fakultät bei?}

\subsection{Erwarteter Gewinn durch die Zusammenarbeit zwischen Klinikerinnen bzw. Klinikern und Grundlagenwissenschaftlerinnen bzw. Grundlagenwissenschaftlern}

\subsection{Erläuterung der leistungsabhängigen Vergabe der Zuführungsbeträge für Forschung und Lehre an der Medizinischen Fakultät}


\section{Koordination}

\subsection{Erläuterung der Umsetzung der gemeinsamen Ziele und des gemeinsamen Arbeitsprogramms im Koordinationsprojekt}

\subsection{Beantragte Module}

\subsubsection{Modul Koordinierung}

\subsubsection{Modul Verbundmittel (Personal-, Sach- und Investitionsmittel)}

\subsubsection{Modul Anschubförderung}

\subsubsection{Modul Chancengleichheitsmaßnahmen in Forschungsverbünden}

\subsubsection{Modul Professur}

\subsubsection{Modul Rotationsstellen}

\subsubsection{Modul Projektspezifische Workshops}

\subsubsection{Modul Mercator-Fellow}

\subsubsection{Modul Öffentlichkeitsarbeit}


\section{Voraussetzungen für die Durchführung des Vorhabens}

\subsection{Angaben zur Dienststellung}

\subsection{Zusammensetzung der Projektarbeitsgruppe}

\subsection{Wissenschaftlerinnen und Wissenschaftler, mit denen für \underline{dieses} Vorhaben eine konkrete Vereinbarung zur Zusammenarbeit besteht}

\subsection{Apparative Ausstattung}

\subsection{Projektrelevante Zusammenarbeit mit erwerbswirtschaftlichen Unternehmen}

\subsection{Projektrelevante Beteiligungen an erwerbswirtschaftlichen Unternehmen}


\section{Ergänzende Erklärungen}

\end{document}
